\input epsf
\magnification\magstephalf
\nopagenumbers
\font\tf=cmr17
\parindent=0pt
\parskip=6pt plus 2pt minus 2pt
\def\Z{\mathop{\rm Z}}%

\topglue1pc
\centerline{\tf Z Curve (Morton) Ordering}
\bigskip
\noindent
Imagine a two-dimensional grid of cells, indexed by integer
$x$ and $y$ coordinates. The cells are numbered in Z curve
order, starting at the lower left corner with $\Z(x,y)=\Z(0,0)=0$
(see figure). The virtue of the Z curve is good locality:
it completes one quadrant before entering the next. Moreover,
it is simple to compute Z values from $x,y$ coordinates and
vice versa by bit interleaving.

The grid's cells can be partitioned into blocks of 2 by 2 cells,
and blocks can be recursively partitioned into larger blocks of
2 by 2 smaller blocks. Cells are said to be blocks at level~0,
blocks of 4 cells are at level~1, blocks of 16 cells at level~2,
and so on up to the one block at level~32 that covers the entire
grid. Blocks are identified by their level $n$ and their lower-left
cell's Z value (see figure).

\nobreak\bigskip
\centerline{\epsfbox{ZCurve.1}}
\medbreak
\noindent
The {\tt ZCurve} utility class provides operations related
to the Z curve, using unsigned 32-bit integers for the
$x$ and $y$ coordinates, and unsigned 64-bit integers
for the interleaved Z values.

$\hbox{\tt Encode}(x,y)$ encodes the $x$ and $y$ coordinates
into a single Z value $z$ by bit interleaving in about
30 bit-level operations (shift, and, or, xor).

$\hbox{\tt DecodeX}(z)$ and $\hbox{\tt DecodeY}(z)$ retrieve
the $x$ and $y$ coordinates from the Z value $z$, each
in about 15 bit-level operations.

$\hbox{\tt IsInsideBox}(z,lo,hi)$ tests if $z$ is within the box
defined by $lo$ and $hi$; instead of decoding the Z values
into x and y components, it masks off every other bit and
compares the resulting 64-bit values.

$\hbox{\tt NextInsideBox}(z,lo,hi)$ finds the least $z'>z$ that
is inside the box defined by $lo$ and $hi$ using the algorithm
described by Tropf and Herzog (``Multidimensional Range Search in
Dynamically Balanced Trees,'' {\sl Angewandte Informatik}, 2/1981).
For example, $\hbox{\tt NextInsideBox}(28,3,49)=33$.

$\hbox{\tt LargestLevel}(z)$ is the largest block level starting
at the given $z$ value; it is $\lfloor\hbox{NTZ}(z)/2\rfloor$
where $\hbox{NTZ}$ is the number of trailing zero bits in $z$.
Overloads find the largest block level starting at a given
$z$ value and not extending beyond a given maximum Z value,
and consider only a given subset of all levels $0\dots32$.
For example, $\hbox{\tt LargestLevel}(0)=32$, but
$\hbox{\tt LargestLevel}(0,49)=2$ and
$\hbox{\tt LargestLevel}(0,49,[1,3])=1$.

$\hbox{\tt GetQueryBlocks}(lo,hi,{\it levels})$ yields the
shortest ordered sequence of blocks covering the whole box
defined by min and max Z values $lo$ and $hi$. The idea is
to use larger blocks inside the box and smaller blocks
along the boundary of the box.
If ${\it levels}$ are provided, only blocks at those levels
are used; in this case, the blocks may extend beyond the box.

\bye
